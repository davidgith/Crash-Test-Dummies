\section{Projektorganisation}
F�r einen reibungslosen Ablauf haben wir zu Beginn geplant, wie wir unsere Kommunikation gestalten wollen und konnten mit Hilfe unten aufgf�hrter Tools eine erfolgsversprechende Roadmap erzielen.

Die Seminarorganisation schrieb ein regelm��iges Treffen mit dem Seminarleiter vor um Hilfestellungen und Tipps zu erm�glichen. Vom Fachgebiet Fahrzeugtechnik wurde alle zwei Wochen eine Fragerunde zu regelungstechnischen Problemen angeboten. Dieses Treffen diente auch als Austausch mit den anderen Gruppen.

In unsere Kleingruppe entschieden wir uns dazu, einmal w�chentlich zusammen am Auto zu arbeiten um gegenseitiges Helfen zu erm�glichen und immer wieder den Stand abzukl�ren sowie den n�chsten kurzen Abschnitt zu planen.

Den Rest der Woche wurde selbstst�ndig an den neu zugeteilten Aufgaben weitergearbeitet.
Bei akuten Fragen und Problemen fand ebenso eine st�ndige Kommunikation �ber WhatsApp statt.

\subparagraph{Trello}
Damit wir die Aufgaben pr�zise und strukturiert festhalten konnten haben wir uns f�r das Organisations-Tool Trello entschieden. Hiermit konnten wir durch Anlegen von Listen unsere Punkte in \glqq ToDo\grqq, \glqq Meeting\grqq, \glqq Gemacht\grqq{} und \glqq Aufgaben f�r die Zukunft\grqq{} aufteilen.

\subparagraph{Github}
F�r die Synchronisierung des Programmcodes haben wir das Versionsverwaltungsprogramm GitHub genutzt. Dadurch konnten wir komfortabel den Programmiercode des Fahrzeugs austauschen sowie bei Fehlern auf alte Code-Zust�nde zur�ckgreifen.

Ebenso konnten wir beim Programmieren auf eine Reihe vorhandener libraries und Funktionen zur�ckgreifen. 

\subparagraph{OpenCV}
Mit OpenCV \cite{OpenCV} konnten wir die Bilder verarbeiten und die ben�tigten Informationen herausfiltern. Auch der Algorithmus zur sign recognition basiert auf einem Beispiel einer OpenCV-Demo \cite{findObject}

\subparagraph{quadprog++}
Die library quadprog++ \cite{quadprog} stellt eine Funktion bereit, welche die Berechnung des MPC erleichtert.

Ebenso haben wir mit dem Programm doxygen \cite{doxygen} eine �bersichtliche Auflistung all unserer Klassen mit Kommentaren erstellt.